\documentclass[11pt,a4paper]{article}
\usepackage[utf8x]{inputenc}
\usepackage{ucs}
\usepackage[spanish]{babel}
\usepackage[left=2cm,top=2cm,right=2cm,bottom=3cm]{geometry} 
\usepackage{amsmath}
\usepackage{amsfonts}
\usepackage{amssymb}
\usepackage{dcolumn}
\usepackage{float}
\usepackage{graphicx}
\usepackage{ esint }
\usepackage{fancyhdr}
\usepackage{enumerate} 
\pagestyle{fancy}
\usepackage{tocbibind}
\usepackage{setspace}
\usepackage{parskip}
\usepackage[hidelinks]{hyperref}
\usepackage{listings} 
\usepackage[svgnames]{xcolor}

\definecolor{codegreen}{rgb}{0,0.6,0}
\definecolor{codegray}{rgb}{0.5,0.5,0.5}
\definecolor{codepurple}{rgb}{0.58,0,0.82}
\definecolor{backcolour}{rgb}{0.95,0.95,0.92}


\definecolor{dkgreen}{rgb}{0,0.6,0}
\definecolor{gray}{rgb}{0.5,0.5,0.5}
\definecolor{mauve}{rgb}{0.58,0,0.82}


\lstset{basicstyle=\ttfamily}
\lstdefinestyle{mystyle}{
	%backgroundcolor=\color{backcolour},   
	commentstyle=\color{codegreen},
	keywordstyle=\color{magenta},
	numberstyle=\tiny\color{codegray},
	stringstyle=\color{codepurple},
	breakatwhitespace=false,         
	breaklines=true,                 
	keepspaces=true,                 
	numbers=left,                    
	%numbersep=5pt                  
}

\lstdefinestyle{customasm}{
  belowcaptionskip=1\baselineskip,
  frame=L,
  xleftmargin=\parindent,
  language= Python,
  basicstyle=\footnotesize\ttfamily,
  commentstyle=\itshape\color{purple!40!black},
}

\lstset{literate=
  {á}{{\'a}}1 {é}{{\'e}}1 {í}{{\'i}}1 {ó}{{\'o}}1 {ú}{{\'u}}1
  {Á}{{\'A}}1 {É}{{\'E}}1 {Í}{{\'I}}1 {Ó}{{\'O}}1 {Ú}{{\'U}}1
  {à}{{\`a}}1 {è}{{\`e}}1 {ì}{{\`i}}1 {ò}{{\`o}}1 {ù}{{\`u}}1
  {À}{{\`A}}1 {È}{{\'E}}1 {Ì}{{\`I}}1 {Ò}{{\`O}}1 {Ù}{{\`U}}1
  {ä}{{\"a}}1 {ë}{{\"e}}1 {ï}{{\"i}}1 {ö}{{\"o}}1 {ü}{{\"u}}1
  {Ä}{{\"A}}1 {Ë}{{\"E}}1 {Ï}{{\"I}}1 {Ö}{{\"O}}1 {Ü}{{\"U}}1
  {â}{{\^a}}1 {ê}{{\^e}}1 {î}{{\^i}}1 {ô}{{\^o}}1 {û}{{\^u}}1
  {Â}{{\^A}}1 {Ê}{{\^E}}1 {Î}{{\^I}}1 {Ô}{{\^O}}1 {Û}{{\^U}}1
  {œ}{{\oe}}1 {Œ}{{\OE}}1 {æ}{{\ae}}1 {Æ}{{\AE}}1 {ß}{{\ss}}1
  {ű}{{\H{u}}}1 {Ű}{{\H{U}}}1 {ő}{{\H{o}}}1 {Ő}{{\H{O}}}1
  {ç}{{\c c}}1 {Ç}{{\c C}}1 {ø}{{\o}}1 {å}{{\r a}}1 {Å}{{\r A}}1
  {€}{{\euro}}1 {£}{{\pounds}}1 {«}{{\guillemotleft}}1
  {»}{{\guillemotright}}1 {ñ}{{\~n}}1 {Ñ}{{\~N}}1 {¿}{{?`}}1
}

\lstset{showstringspaces=false}
\lstloadlanguages{Python}
\lstset{basicstyle=\ttfamily\footnotesize}
\lstset{style=mystyle}
\usepackage[titletoc,toc,page]{appendix}
\usepackage{pdfpages}
\renewcommand{\appendixtocname}{Anexo}
\renewcommand{\appendixpagename}{Anexo}
\lhead{ Modelos y Simulación- TP1 }
\rhead{\includegraphics[width=1.5 cm]{imagenes/logo}}
\author{cyn}
\begin{document}
\begin{titlepage}
\begin{center}
\vspace*{-1in}
\begin{figure}[htb]
\begin{flushleft}
\includegraphics[width=5cm]{imagenes/logo}
\end{flushleft}
\end{figure}
\begin{LARGE}
\textbf{U.B.A. FACULTAD DE INGENIERÍA}\\
\end{LARGE}
\vspace*{0.15in}
\begin{LARGE}
\textbf{Departamento de Computación}\\
\end{LARGE}
\vspace*{0.2in}
\begin{LARGE}
\textbf{Modelos y Simulación 7526 - 9519}\\
\end{LARGE}
\vspace*{0.2in}
\begin{Large}
\textbf{TRABAJO PRÁCTICO \#1}\\
\end{Large}
\vspace*{0.2in}
\begin{LARGE}
\textit{Números al azar y Test estadísticos }\\
\end{LARGE}
\vspace*{0.2in}
\begin{Large}
\raggedright\textbf{Curso: 2019 - 1er Cuatrimestre}\\
\end{Large}
\vspace*{0.1in}
\begin{Large}
\raggedright\textbf{Turno: Miércoles}\\
\end{Large}
\vspace*{0.1in}

\begin{table}[htb]
\begin{center}
\begin{spacing}{1.9}
\begin{tabular}{| l | l |}
\hline
\multicolumn{2}{|>{\arraybackslash}p{15cm}|}{\begin{Large}
\textbf{GRUPO N° 1}
\end{Large}}\\
\hline
\textbf{Integrantes} & \textbf{Padrón} \\
\hline
\makebox[8cm][c]{Amurrio, Gastón} & \makebox[2.5cm][c]{*****}\\
\hline
\makebox[8cm][c]{Gamarra Silva, Cynthia Marlene} & \makebox[2.5cm][c]{92702}\\
\hline
\makebox[8cm][c]{****, *****} & \makebox[2.5cm][c]{*****}\\
\hline
\textbf{Fecha de Entrega: } & \hspace{0.8cm}24-04-2019\\
\hline
\textbf{Fecha de aprobación: } & \\
\hline
\textbf{Calificación: } & \\
\hline
\textbf{Firma de aprobación:} & \\
\hline
\end{tabular}
\end{spacing}
\end{center}
\end{table}
\fbox{%
\begin{minipage}[c][3.4cm][l]{.9\linewidth}
\textbf{Observaciones:} \\
\vfill
\end{minipage}
}
\end{center}

\vspace*{0.1in}
\end{titlepage}
\tableofcontents 
\vspace*{0.3in}
\newpage

\includepdf[pages=1,scale=0.95,pagecommand = \section{Enunciado del trabajo práctico}\label{enunciado},offset=10 -10]{FIUBA-Simulacion-TrabajoPractico1.pdf}
\includepdf[pages={2-last},scale=0.95,pagecommand = {},offset=10 -10]{FIUBA-Simulacion-TrabajoPractico1.pdf}


\newpage

\section{Introducción}

\section{Conceptos téoricos}


\section{Diseño e implementación}
Para cada uno de los ejercicios pedidos se realiza una explicación de cada uno de ellos. Se toma como base teórica lo explicado en la sección anterior.

	\subsection{Ejercicio 1}


	\subsection{Ejercicio 2}


	\subsection{Ejercicio 3}


	\subsection{Ejercicio 4}


	\subsection{Ejercicio 5}


	\subsection{Ejercicio 6}


	\subsection{Ejercicio 7}


	\subsection{Ejercicio 8}


	\subsection{Ejercicio 9}


	\subsection{Ejercicio 10}

\newpage
\section{Conclusiones}
El trabajo práctico nos permitió conocer y realizar simulaciones teniendo como base teórica los conceptos explicados en clase . Además, nos permitió conocer herramientas que permiten realizar simulaciones qure son muy utilizadas en el campo científico.


\begin{thebibliography}{10}
	\bibitem{} Python, Generación de números con distintas distribuciones de probabilidad, https://docs.python.org/3/library/random.html.
\end{thebibliography}

\newpage
%-----------------------------------%
%									%
%			Seccion:Fuente			%
%									%
%-----------------------------------%
\appendix
\section{Código fuente}\label{appendix_codigo_fuente}

	\subsection{Resolución ejercicio 1}\label{ejercicio_1}
		\lstinputlisting[language=Python]{../ejercicios/TPEj1.py}

	\newpage

	\subsection{Resolución ejercicio 2}\label{ejercicio_2}
		\lstinputlisting[language=Python]{../ejercicios/TPEj2.py}

	\newpage

	\subsection{Resolución ejercicio 3}\label{ejercicio_3}
		\lstinputlisting[language=Python]{../ejercicios/TPEj3.py}

	\newpage

	\subsection{Resolución ejercicio 4}\label{ejercicio_4}
		\lstinputlisting[language=Python]{../ejercicios/TPEj4.py}

	\newpage

	\subsection{Resolución ejercicio 5}\label{ejercicio_5}
		\lstinputlisting[language=Python]{../ejercicios/TPEj5.py}

	\newpage

	\subsection{Resolución ejercicio 6}\label{ejercicio_6}
		\lstinputlisting[language=Python]{../ejercicios/TPEj6.py}

	\newpage

	\subsection{Resolución ejercicio 7}\label{ejercicio_7}
		\lstinputlisting[language=Python]{../ejercicios/TPEj7.py}

	\newpage

	\subsection{Resolución ejercicio 8}\label{ejercicio_8}
		\lstinputlisting[language=Python]{../ejercicios/TPEj8.py}

	\newpage

	\subsection{Resolución ejercicio 9}\label{ejercicio_9}
		\lstinputlisting[language=Python]{../ejercicios/TPEj9.py}

	\newpage

	\subsection{Resolución ejercicio 10}\label{ejercicio_10}
		\lstinputlisting[language=Python]{../ejercicios/TPEj10.py}

\end{document}
