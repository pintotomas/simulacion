\documentclass[11pt,a4paper]{article}
\usepackage[utf8x]{inputenc}
\usepackage{ucs}
\usepackage[spanish]{babel}
\usepackage[left=2cm,top=2cm,right=2cm,bottom=3cm]{geometry} 
\usepackage{amsmath}
\usepackage{amsfonts}
\usepackage{amssymb}
\usepackage{dcolumn}
\usepackage{float}
\usepackage{graphicx}
\usepackage{ esint }
\usepackage{fancyhdr}
\usepackage{enumerate} 
\pagestyle{fancy}
\usepackage{tocbibind}
\usepackage{setspace}
\usepackage{parskip}
\usepackage[hidelinks]{hyperref}
\usepackage{listings} 
\usepackage[svgnames]{xcolor}

\definecolor{codegreen}{rgb}{0,0.6,0}
\definecolor{codegray}{rgb}{0.5,0.5,0.5}
\definecolor{codepurple}{rgb}{0.58,0,0.82}
\definecolor{backcolour}{rgb}{0.95,0.95,0.92}


\definecolor{dkgreen}{rgb}{0,0.6,0}
\definecolor{gray}{rgb}{0.5,0.5,0.5}
\definecolor{mauve}{rgb}{0.58,0,0.82}


\lstset{basicstyle=\ttfamily}
\lstdefinestyle{mystyle}{
	%backgroundcolor=\color{backcolour},   
	commentstyle=\color{codegreen},
	keywordstyle=\color{magenta},
	numberstyle=\tiny\color{codegray},
	stringstyle=\color{codepurple},
	breakatwhitespace=false,         
	breaklines=true,                 
	keepspaces=true,                 
	numbers=left,                    
	%numbersep=5pt                  
}

\lstdefinestyle{customasm}{
  belowcaptionskip=1\baselineskip,
  frame=L,
  xleftmargin=\parindent,
  language= Python,
  basicstyle=\footnotesize\ttfamily,
  commentstyle=\itshape\color{purple!40!black},
}

\lstset{literate=
  {á}{{\'a}}1 {é}{{\'e}}1 {í}{{\'i}}1 {ó}{{\'o}}1 {ú}{{\'u}}1
  {Á}{{\'A}}1 {É}{{\'E}}1 {Í}{{\'I}}1 {Ó}{{\'O}}1 {Ú}{{\'U}}1
  {à}{{\`a}}1 {è}{{\`e}}1 {ì}{{\`i}}1 {ò}{{\`o}}1 {ù}{{\`u}}1
  {À}{{\`A}}1 {È}{{\'E}}1 {Ì}{{\`I}}1 {Ò}{{\`O}}1 {Ù}{{\`U}}1
  {ä}{{\"a}}1 {ë}{{\"e}}1 {ï}{{\"i}}1 {ö}{{\"o}}1 {ü}{{\"u}}1
  {Ä}{{\"A}}1 {Ë}{{\"E}}1 {Ï}{{\"I}}1 {Ö}{{\"O}}1 {Ü}{{\"U}}1
  {â}{{\^a}}1 {ê}{{\^e}}1 {î}{{\^i}}1 {ô}{{\^o}}1 {û}{{\^u}}1
  {Â}{{\^A}}1 {Ê}{{\^E}}1 {Î}{{\^I}}1 {Ô}{{\^O}}1 {Û}{{\^U}}1
  {œ}{{\oe}}1 {Œ}{{\OE}}1 {æ}{{\ae}}1 {Æ}{{\AE}}1 {ß}{{\ss}}1
  {ű}{{\H{u}}}1 {Ű}{{\H{U}}}1 {ő}{{\H{o}}}1 {Ő}{{\H{O}}}1
  {ç}{{\c c}}1 {Ç}{{\c C}}1 {ø}{{\o}}1 {å}{{\r a}}1 {Å}{{\r A}}1
  {€}{{\euro}}1 {£}{{\pounds}}1 {«}{{\guillemotleft}}1
  {»}{{\guillemotright}}1 {ñ}{{\~n}}1 {Ñ}{{\~N}}1 {¿}{{?`}}1
}

\lstset{showstringspaces=false}
\lstloadlanguages{Python}
\lstset{basicstyle=\ttfamily\footnotesize}
\lstset{style=mystyle}
\usepackage[titletoc,toc,page]{appendix}
\usepackage{pdfpages}
\renewcommand{\appendixtocname}{Anexo}
\renewcommand{\appendixpagename}{Anexo}
\lhead{ Modelos y Simulación- TP1 }
\rhead{\includegraphics[width=1.5 cm]{imagenes/logo}}
\author{cyn}
\begin{document}
\begin{titlepage}
\begin{center}
\vspace*{-1in}
\begin{figure}[htb]
\begin{flushleft}
\includegraphics[width=5cm]{imagenes/logo}
\end{flushleft}
\end{figure}
\begin{LARGE}
\textbf{U.B.A. FACULTAD DE INGENIERÍA}\\
\end{LARGE}
\vspace*{0.15in}
\begin{LARGE}
\textbf{Departamento de Computación}\\
\end{LARGE}
\vspace*{0.2in}
\begin{LARGE}
\textbf{Modelos y Simulación 7526 - 9519}\\
\end{LARGE}
\vspace*{0.2in}
\begin{Large}
\textbf{TRABAJO PRÁCTICO \#1}\\
\end{Large}
\vspace*{0.2in}
\begin{LARGE}
\textit{Números al azar y Test estadísticos }\\
\end{LARGE}
\vspace*{0.2in}
\begin{Large}
\raggedright\textbf{Curso: 2019 - 1er Cuatrimestre}\\
\end{Large}
\vspace*{0.1in}
\begin{Large}
\raggedright\textbf{Turno: Miércoles}\\
\end{Large}
\vspace*{0.1in}

\begin{table}[htb]
\begin{center}
\begin{spacing}{1.9}
\begin{tabular}{| l | l |}
\hline
\multicolumn{2}{|>{\arraybackslash}p{15cm}|}{\begin{Large}
\textbf{GRUPO N° 1}
\end{Large}}\\
\hline
\textbf{Integrantes} & \textbf{Padrón} \\
\hline
\makebox[8cm][c]{Amurrio, Gastón} & \makebox[2.5cm][c]{93584}\\
\hline
\makebox[8cm][c]{Gamarra Silva, Cynthia Marlene} & \makebox[2.5cm][c]{92702}\\
\hline
\makebox[8cm][c]{Pinto, Tomás} & \makebox[2.5cm][c]{98757}\\
\hline
\textbf{Fecha de Entrega: } & \hspace{0.8cm}24-04-2019\\
\hline
\textbf{Fecha de aprobación: } & \\
\hline
\textbf{Calificación: } & \\
\hline
\textbf{Firma de aprobación:} & \\
\hline
\end{tabular}
\end{spacing}
\end{center}
\end{table}
\fbox{%
\begin{minipage}[c][3.4cm][l]{.9\linewidth}
\textbf{Observaciones:} \\
\vfill
\end{minipage}
}
\end{center}

\vspace*{0.1in}
\end{titlepage}
\tableofcontents 
\vspace*{0.3in}
\newpage

\includepdf[pages=1,scale=0.95,pagecommand = \section{Enunciado del trabajo práctico}\label{enunciado},offset=10 -10]{FIUBA-Simulacion-TrabajoPractico1.pdf}
\includepdf[pages={2-last},scale=0.95,pagecommand = {},offset=10 -10]{FIUBA-Simulacion-TrabajoPractico1.pdf}


\newpage

\section{Introducción}
El trabajo práctico consiste en aplicar conceptos teóricos explicados en clase sobre generación de números aleatorios aplicado a distintos métodos estadísticos utilizados en el medio ciéntifico como ser Box Muller, Generador Congruencial Lineal (GCL), Tranformada inversa y tests como Test espectral y Kolmogorov-Smirnov.
Los ejercicios están simulados en lenguaje Python.

\section{Implementación y resultados}
Para cada uno de los ejercicios pedidos se realiza una explicación de cada uno de ellos. Se toma como base teórica lo explicado en clase tanto teórica como clase práctica.

	\subsection{Ejercicio 1}
		El resultado de los primeros 5 números de la secuencia: [62978, 383030987L, 2740587618L, 1650525291L, 2470812354L]\\
		El histograma pedido utilizando el método Generador Congruencial Lineal (GCL) donde se grafica para números al azar entre 0 y 1,es el siguiente:
		\begin{figure}[H]
  			\centering
    			\includegraphics[width=14cm]{imagenes/histogramaEjer1}
		\end{figure}

	\subsection{Ejercicio 2}
		El histograma pedido utilizando el método de la transformada inversa generado con números pseudoaleatorios con distribución exponencial negativa de media 20 es el siguiente:
		\begin{figure}[H]
  			\centering
    			\includegraphics[width=14cm]{imagenes/histogramaEjer2}
		\end{figure}
		Comparando los resultados simulados y teóricos:
		\begin{itemize}
			\item El valor simulado de la media es 0.0501097366318
			\item El valor teórico de la media es 0.05
			\item El valor simulado de la varianza es 0.00250751904958
			\item El valor teórico de la varianza es 0.0025
		\end{itemize}
		
		Por lo tanto podemos observar que los valores simulados y teóricos son bastantes parecidos.

	\subsection{Ejercicio 3}

		Los resultados que obtenemos son los siguientes:
		\begin{itemize}
			\item El valor simulado de la media z1 es 0.000743926097041
			\item El valor simulado de la media z2 es -0.0016462929471
			\item El valor teórico de la media es 0 
			\item El valor simulado de la varianza z1 es 0.998271254875
			\item El valor simulado de la varianza z2 es 0.996519878651
			\item El valor teórico de la varianza es 1
		\end{itemize}

		El histograma pedido utilizando Box Muller es el siguiente:
		\begin{figure}[H]
  			\centering
    			\includegraphics[width=14cm]{imagenes/histogramaEjer3}
		\end{figure}

		Si comparamos con una distribución Normal estándar obtenemos:
		\begin{figure}[H]
  			\centering
    			\includegraphics[width=20cm]{imagenes/histogramasEjer3}
		\end{figure}
		
		Por lo tanto podemos observar que se comprueba que el método de Box Muller es una distribución normal estándar.


	\subsection{Ejercicio 4}


	\subsection{Ejercicio 5}
		El histograma pedido utilizando la función de distribución de probabilidad empírica dada por el enunciado:
		\begin{figure}[H]
  			\centering
    			\includegraphics[width=14cm]{imagenes/histogramaEjer5}
		\end{figure}

	\subsection{Ejercicio 6}
		El gráfico pedido utilizando tilizando una distribucion uniforme entre [-1,1] generado con números aleatorios en un círculo de radio 1 centrado en el origen.
		\begin{figure}[H]
  			\centering
    			\includegraphics[width=14cm]{imagenes/histogramaEjer6}
		\end{figure}

	\subsection{Ejercicio 7}


	\subsection{Ejercicio 8}
	Utilizando el test Estadístico $Chi^2$ se aplica el método utilizando los siguientes pasos:
	\begin{enumerate}
		\item Utilizamos la distribución empírica generada en el ejercicio 5). Está distribución se corresponderá a $N_i$ ocurrencias observadas.
		\item Obtenemos una distribución uniforme con n = 100.000 muestras generadas
		\item Medimos la dispersion de las ocurrencias obervadas $N_i$ respectos de las esperadas $n*p_i$. La dispersión se calcula como:
		\begin{align*}
		D^2 = \Sigma_{k=1}^{k-1} \frac{(N_i - np_i)^2}{np_i}
		\end{align*}
		\item Calculamos $D^2 < t$ para saber si aceptamos la hipótesis  con un error del 1\%
		
		En nuestro caso obtuvimos que se rechaza la hipótesis con la distribución empirica con un error del 1 \%
	\end{enumerate}

	\subsection{Ejercicio 9}


	\subsection{Ejercicio 10}

\newpage
\section{Conclusiones}
El trabajo práctico nos permitió conocer y realizar simulaciones teniendo como base teórica los conceptos explicados en clase . Además, nos permitió conocer herramientas que permiten realizar simulaciones qure son muy utilizadas en el campo científico.


\begin{thebibliography}{10}
	\bibitem{} Python, Generación de números con distintas distribuciones de probabilidad, https://docs.python.org/3/library/random.html.
	\bibitem{} Método de Box Muller, https://es.wikipedia.org/wiki/Método\_de\_Box-Muller.
	\bibitem{} Probability, Statistics, and Random Processes for Electrical Engineerging,3rd\_Ed. Leon-Garcia
\end{thebibliography}


\newpage
%-----------------------------------%
%									%
%			Seccion:Fuente			%
%									%
%-----------------------------------%
\appendix
\section{Código fuente}\label{appendix_codigo_fuente}

	\subsection{Resolución ejercicio 1}\label{ejercicio_1}
		\lstinputlisting[language=Python]{../ejercicios/TPEj1.py}

	\newpage

	\subsection{Resolución ejercicio 2}\label{ejercicio_2}
		\lstinputlisting[language=Python]{../ejercicios/TPEj2.py}

	\newpage

	\subsection{Resolución ejercicio 3}\label{ejercicio_3}
		\lstinputlisting[language=Python]{../ejercicios/TPEj3.py}

	\newpage

	\subsection{Resolución ejercicio 4}\label{ejercicio_4}
		\lstinputlisting[language=Python]{../ejercicios/TPEj4.py}

	\newpage

	\subsection{Resolución ejercicio 5}\label{ejercicio_5}
		\lstinputlisting[language=Python]{../ejercicios/TPEj5.py}

	\newpage

	\subsection{Resolución ejercicio 6}\label{ejercicio_6}
		\lstinputlisting[language=Python]{../ejercicios/TPEj6.py}

	\newpage

	\subsection{Resolución ejercicio 7}\label{ejercicio_7}
		\lstinputlisting[language=Python]{../ejercicios/TPEj7.py}

	\newpage

	\subsection{Resolución ejercicio 8}\label{ejercicio_8}
		\lstinputlisting[language=Python]{../ejercicios/TPEj8.py}

	\newpage

	\subsection{Resolución ejercicio 9}\label{ejercicio_9}
		\lstinputlisting[language=Python]{../ejercicios/TPEj9.py}

	\newpage

	\subsection{Resolución ejercicio 10}\label{ejercicio_10}
		\lstinputlisting[language=Python]{../ejercicios/TPEj10.py}

\end{document}
